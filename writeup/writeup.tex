% 
% Annual Cognitive Science Conference
% Sample LaTeX Paper -- Proceedings Format
% 

% Original : Ashwin Ram (ashwin@cc.gatech.edu)       04/01/1994
% Modified : Johanna Moore (jmoore@cs.pitt.edu)      03/17/1995
% Modified : David Noelle (noelle@ucsd.edu)          03/15/1996
% Modified : Pat Langley (langley@cs.stanford.edu)   01/26/1997
% Latex2e corrections by Ramin Charles Nakisa        01/28/1997 
% Modified : Tina Eliassi-Rad (eliassi@cs.wisc.edu)  01/31/1998
% Modified : Trisha Yannuzzi (trisha@ircs.upenn.edu) 12/28/1999 (in process)
% Modified : Mary Ellen Foster (M.E.Foster@ed.ac.uk) 12/11/2000
% Modified : Ken Forbus                              01/23/2004
% Modified : Eli M. Silk (esilk@pitt.edu)            05/24/2005
% Modified: Niels Taatgen (taatgen@cmu.edu)  10/24/2006

%% Change ``a4paper'' in the following line to ``letterpaper'' if you are
%% producing a letter-format document.


\documentclass[10pt,letterpaper]{article}

\usepackage{cogsci}
\usepackage{pslatex}
\usepackage{apacite}
\usepackage{graphicx}
\usepackage{color}
\usepackage{amsmath}
\usepackage{multirow}
\usepackage{amssymb}
%\usepackage{url}
%\usepackage{hyperref}


 
\definecolor{Red}{RGB}{255,0,0}
\newcommand{\red}[1]{\textcolor{Red}{#1}}  

\title{ Bodies in (e)motion: \\ Human and machine inference of emotions from dance kinematics \\ \textit{author order can be changed} }
 
\author{{\large \bf Desmond C. Ong (dco@stanford.edu)} \\
  Department of Psychology, Stanford University, Stanford CA, USA 
  \And {\large \bf Carolyn Fu (...)} \\
  Affiliation
}

\begin{document}

\maketitle

\begin{abstract}
abstract

\textbf{Keywords:} 
keywords
\end{abstract}

Writing




%%%%

emotion from dance movement \cite{Camurri2003, Camurri2004}

recognition of emotions from gestures \cite{Atkinson2007}

arm movement \cite{Pollick2001}
face and body gestures \cite{Gunes2007}

model \cite{Schindler2008, Shikanai2013}

\section{Section}

neurobiology of emotional body language \cite{deGelder2006} 

dancers watching dancers \cite{Cross2006}

people with dance experience report greater affective evaluation from watching dance \cite{Kirsch2013}

\subsubsection{Participants and Procedures.} 

\subsubsection{Results.} 

\begin{figure}[htb!]
%\begin{center}\includegraphics[width=1\columnwidth]{images/Expt1results.png}\end{center}
\caption{ Expt 1 Results. }
\label{Expt1ResultFig}
\end{figure}



\begin{table}
\scalebox{0.7}{
\begin{tabular}{l|c|c}
X & X & X \\
\hline
X & \multirow{2}{*}{\checkmark \checkmark} & \multirow{2}{*}{\checkmark}\\
X & & \\
\hline
X & \multirow{2}{*}{\checkmark} & \multirow{2}{*}{\checkmark \checkmark} \\
X & & \\
\hline
X & \multirow{2}{*}{-}  & \multirow{2}{*}{-} \\
X & &
\end{tabular}
}
\caption{Predictions for Experiment 2. }
\label{Expt2Predictions}
\end{table}


\section{Discussion}


\section{Acknowledgments}

This work was supported in part by an A*STAR National Science Scholarship to DCO and ...

\bibliographystyle{apacite}

\setlength{\bibleftmargin}{.125in}
\setlength{\bibindent}{-\bibleftmargin}

\bibliography{writeup}


\end{document}
